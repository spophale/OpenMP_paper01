To understand the relationship between OpenMP affinity and data locality in the POWER8 architecture we want to quantify the effect of data locality under different combinations of OMP\_PROC\_BIND and OMP\_PLACES settings and quantify their effects using the available POWER8 hardware counter information.

\subsection{Experimental Setup}
\subsubsection{Test System}
Four our experiments we use a dual socket POWER8 system with 256GB of main memory. This system contains four POWER8 chiplets (two on each socket) that map to  four NUMA nodes. The system contains a total of 20 cores over two sockets (10 each) that have the capacity to runt 8 hardware threads per core, thus providing a total of up to 160 threads for computation. 
% four NVIDIA Tesla K40m GPUs, two Mellanox Connect-IB InfiniBand FDR (56 Gb/s) ports and one 4-port Gigabit Ethernet switch. 

\subsubsection{The Experiment}
To able to identify the correlation between OpenMP affinity features, performance and the hardware counters, we use the Jacobi iterative method program to solve a 
finite difference discretization of Helmholtz equation (here on referred to as \textit{Jacobi program}). We experiment with the locality of the data by controlling the initialization of the parallel loop at the start of the program. 
When all threads initialize the sections of data arrays in parallel using the omp\_parallel construct, this allows for the 
\textit{first-touch} policy described in Section~\ref{sec:intro} to place data closer to the physical CPU executing the OpenMP thread thus resulting in good data locality. 
We use the \textit{num\_threads(1)} so that only the one thread will initialize the data causing data to be placed only near the physical NUMA domain executing the single thread. This results in bad data locality.
We then test combinations of OpenMP \textit{bindings} and \textit{places} for both these versions to compare and contrast their performance and the values of the hardware counters mentioned in Table~\ref{tab:hwct}. Based on this information we want to measure the effect of data locality under different OpenMP affinity setting. We also demonstrate the use of the POWER8 hardware counters to measure the data placement effects on the memory subsystem.

%The ultimate aim of this experiment is to be able to \textit{suggest} the correct combination of OpenMP \textit{bindings} and \textit{places} based on 
%either the range, ratio or value of the different hardware counters.