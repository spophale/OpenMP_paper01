The IBM POWER8 system has two configurations of either 6, 10 or 12 cores per processor chip. A typical 12 core processor has  512 KB SRAM L2 caches per core, 96 MB eDRAM shared L3 and an off chip L4 cache that provides up to 128 MB eDRAM space. This is a big improvement over the POWER7 processor. We look closely at the different hardware counters available on the new POWER8 system to better understand the effects of data affinity and thread placement. Performance instrumentation in POWER8 is provided in two layers: the \textbf{Core Level Performance Monitoring} and the \textbf{Nest Level Performance Monitoring}. a number of hardware counters to monitor application characteristics over the entire run. We first look at these before we try to dissect the application. The primary hardware counters we focus on are listed in table~\ref{tab:hwct}. Here we focus on the stall cycles to better understand the overhead of using different \textit{bind} and \textit{place} configurations. The collection of the counter values are enabled by a system provided script. This allows for access to counters that may not be represented as literary strings and accessible via other . 

\begin{table*}[t]
\vspace{-0.5pc}
\caption{POWER8 Relevant Counters}
\centering
\begin{tabular} { | c | c | }
\hline
{\bf Hardware Counter} & {\bf Description}  \\ \hline
PM\_CMPLU\_STALL\_DCACHE\_MISS & Stall cycles due to data cache misses	\\	\hline
PM\_CMPLU\_STALL\_DMISS\_REMOTE & Stall cycles due to data cache miss on remote \\	\hline
PM\_CMPLU\_STALL\_DMISS\_DISTANT & Stall cycles due to data cache miss on distant \\ \hline
\end{tabular}
\label{tab:hwct}
%\vspace{-0.0pc}
\end{table*}
