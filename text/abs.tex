As we move toward pre-Exascale systems, two of the DOE leadership class systems will consist of very powerful OpenPOWER compute nodes which which will be more complex to program. These will have massive amounts of parallelisms where threads may be running on Power9 cores as well as on accelerators. Advances in memory interconnects, such as NVLINK, will provide a unified shared memory address spaces for different types of memories HBM, DRAM, etc. In preparation for such system, we need to improve our understanding on how OpenMP supports the concept of affinity as well as memory placement on POWER8 systems, and if this work needs to be extended to support heterogenous systems that will have massive amounts of threads. Data locality and affinity are key 
program optimizations to exploit the compute and memory capabilities to achieve good performance by minimizing data motion across NUMA domains and access the cache efficiently. This  paper is the first step to evaluate the current features of OpenMP 4.0 on the POWER8 processors, and on how to measure its effects on a system with two POWER8 sockets. We experiment with the different affinity settings provided by OpenMP 4.0 affinity to quantify the costs of having good data locality vs not,  and measure their effects via hardware counters. We also find out which affinity settings benefits more from data locality. Based on this study we describe the current state of art, the challenges we faced in quantifying effects of affinity, and and ideas on how OpenMP 5.0 should be improved to address affinity in the context of NUMA domains and accelerators.

