In this paper we evaluated OpenMP affinity support as well as memory placement on the POWER8 architecture. 
Data locality and affinity are key concepts to exploit the compute and memory capabilities to achieve good performance by minimizing data motion across NUMA domains. 
The main contribution of this paper is to evaluate current affinity features of OpenMP 4.0 on the POWER8 processors, and on how to measure its effects on data locality on a system with two P8 sockets. 
%What we observe from the experiments is that the effect of data placement and data locality is dependent on how threads are mapped to the architectures. 
In some OpenMP affinity test cases, we show that the POWER8 architecture, using its NUCA L3 caches, can hide the cost of accessing remote memory (e.g, as shown in the experiment \textit{(spread, cores)} when running a thread per core since maximizes local caches are available per thread. 
In other cases, when threads share some of the cores, there is a benefit of cache reuse in the non-shared L2 and L1 caches thus improving data locality in the application.
 In this paper we show that optimizing an application for data locality, the improvements will depend on the kind of affinity used. 

Future version of OpenMP affinity model need to support better the concept of NUMA domains. 
This is possibly another \textit{place} option called \textit{Numa} can be added to OpenMP 5.0 via the \emph{OMP\_PLACES} so that threads can be mapped more efficiently to NUMA domains. Although this can be achieved by using OS supported bindings (\textit{taskset} on linux platforms), it is not a portable mechanism. By introducing the support of NUMA domains at the OpenMP level, we can keep the implementation details opaque from the programmer while providing a portable solution across all architectures.


Another type of extensions is to integrate the concept of OMP\_PLACES with the OpenMP target directive and device\_num. Our next step would be to explore ways of mapping OpenMP \emph{target} and \emph{target data} 
regions to NUMA domains to control data and thread placement.
