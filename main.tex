
% This is LLNCS.DEM the demonstration file of
% the LaTeX macro package from Springer-Verlag
% for Lecture Notes in Computer Science,
% version 2.2 for LaTeX2e
%
\documentclass{llncs}
%%%%%%%%%%%%%%%%%%%%%%%%%%%%
% *** MISC UTILITY PACKAGES ***
%
\usepackage{ifpdf}

% *** CITATION PACKAGES ***
\usepackage{cite}

% *** GRAPHICS RELATED PACKAGES ***
%
\usepackage[pdftex]{graphicx}

% *** MATH PACKAGES ***
\usepackage[cmex10]{amsmath}

% *** SPECIALIZED LIST PACKAGES ***
\usepackage{algorithmic}

%SP: Added to reduce verticle space between the subsections
\setlength{\topskip}{0pt}

% *** ALIGNMENT PACKAGES ***
\usepackage{array}
\usepackage{mdwmath}
\usepackage{mdwtab}
%\usepackage{eqparbox}

% *** SUBFIGURE PACKAGES ***
%\usepackage[tight,footnotesize]{subfigure}
\usepackage{caption}
\usepackage{subcaption}

% *** FLOAT PACKAGES ***
\usepackage{fixltx2e}

%\usepackage{stfloats}
% stfloats.sty was written by Sigitas Tolusis. This package gives LaTeX2e
% the ability to do double column floats at the bottom of the page as well
% as the top. (e.g., "\begin{figure*}[!b]" is not normally possible in
% LaTeX2e). It also provides a command:
%\fnbelowfloat
% to enable the placement of footnotes below bottom floats 

% *** PDF, URL AND HYPERLINK PACKAGES ***
\usepackage{url}

% correct bad hyphenation here
\hyphenation{Shar-ed Memo-ry Pass-ing  Mess-age Be-sides bench-mark su-pports uni-versity Carolina}

\usepackage{flushend}
\usepackage{listings}
\usepackage{color}
\usepackage{framed}  %package for framing images
\usepackage{lscape}
\usepackage{hyperref}
%\usepackage{subfig}
\usepackage{float}
\usepackage{multirow}

%\usepackage[options]{algorithm}

%\floatstyle{boxed} 
\restylefloat{figure}

\setcounter{secnumdepth}{5}
% NOTE:
%\section{} % level 1
%\subsection{}  %level 2
%\subsubsection{} % level 3
%\paragraph{} % level 4 - equivalent to subsubsubsection
%\subparagraph{} % level 5
% 
\definecolor{dkgreen}{rgb}{0,0.6,0}
\definecolor{gray}{rgb}{0.5,0.5,0.5}
\definecolor{mauve}{rgb}{0.58,0,0.82}
 
\lstset{ 
  language=C,                     % the language of the code
  %basicstyle=\footnotesize, 
  basicstyle=\ttfamily\scriptsize,      % the size of the fonts that are used for the code
  numbers=left,                   % where to put the line-numbers
  numberstyle=\tiny\color{gray},  % the style that is used for the line-numbers
  stepnumber=1,                   % the step between two line-numbers. If it's 1, each line 
                                    % will be numbered 
  backgroundcolor=\color{white},  % choose the background color. You must add \usepackage{color}
  showspaces=false,               % show spaces adding particular underscores
  showstringspaces=false,         % underline spaces within strings
  showtabs=false,                 % show tabs within strings adding particular underscores
  rulecolor=\color{black},        % if not set, the frame-color may be changed on line-breaks within not-black text (e.g. commens (green here))
  tabsize=1,                      % sets default tabsize to 2 spaces
  %captionpos=b,                   % sets the caption-position to bottom
  breaklines=true,                % sets automatic line breaking
  breakatwhitespace=true,        % sets if automatic breaks should only happen at whitespace
  %title=\lstname,                 % show the filename of files included with \lstinputlisting;
  caption=\lstname,                                 % also try caption instead of title
  keywordstyle=\color{blue},          % keyword style
  commentstyle=\color{dkgreen},       % comment style
  stringstyle=\color{mauve},         % string literal style
  escapeinside={\%*}{*)},            % if you want to add LaTeX within your code
  xleftmargin=15pt,
  morekeywords={*,...}              % if you want to add more keywords to the set
}
%%%%%%%%%%%%%%%%%%%%%%%%%%%%
\usepackage{makeidx}  % allows for indexgeneration
%
\lstset{numbers=left,numberblanklines=false,escapeinside=||}
\let\origthelstnumber\thelstnumber
\makeatletter
\newcommand*\Suppressnumber{%
  \lst@AddToHook{OnNewLine}{%
    \let\thelstnumber\relax%
     \advance\c@lstnumber-\@ne\relax%
    }%
}

\newcommand*\Reactivatenumber{%
  \lst@AddToHook{OnNewLine}{%
   \let\thelstnumber\origthelstnumber%
   \advance\c@lstnumber\@ne\relax}%
}
%%%%%%%%%%%%%%%%%%%%%%%%%%%%%%%%%%%%%%%%%%%%%%%%%%%%%%%%%%%%%
\begin{document}
%
\frontmatter          % for the preliminaries
%
\pagestyle{headings}  % switches on printing of running heads
%\mainmatter              % start of the contributions
%%
\title{Working Title: Exploring OpenMP Affinity on Power8 Architecture }

%Affinity evaluation in OpenMP 4.5  
%Affinity and accelerators - using device_num, data regions for the accelerator, however none of these mechanism are aware of NUMA domains.
%Using the OpenMP 4.5 target data directive it could be a mechanism to control affinity
%
%Places in the OpenMP 4.5 we need a new way to specify a NUMA domain in the context of a device. 
%
%#pragma omp target data device(host:NUMA1) map(tofrom: a, b: zerocopy)  
%{
%
%#pragma omp target device(host:NUMA1) 
%pragma omp parallel 
% // these threads belong to the NUMA node.  
%  // NUMA domain
%
%}



%
%\titlerunning{Hamiltonian Mechanics}  % abbreviated title (for running head)
%                                     also used for the TOC unless
%                                     \toctitle is used
%
\author{Swaroop Pophale \and Oscar Hernandez
}
%
%%%% modified list of authors for the TOC (add the affiliations)
\tocauthor{Swaroop Pophale, Oscar Hernandez (Oak Ridge National Laboratory)}
%
\institute{
Oak Ridge National Laboratory, Oak Ridge, Tennessee, 37840, USA\\
\email{pophaless,oscar@ornl.gov}
}
\maketitle              % typeset the title of the contribution

\begin{abstract}
As we move toward Exascale, some of the OpenPOWER compute nodes for these systems 
are becoming more powerful and complex to program. In the next generation systems will have massive amounts of parallelisms where threads 
may be running on CPU cores as well as on accelerators. Advances in memory interconnects, such as NVLINK, will provide a 
shared memory address spaces for different types of memories including high bandwidth memory (HBM), DRAM, etc. 
In preparation for such systems, we need to work more on our understanding of current in-node programming models like OpenMP supports
the concept of affinity as well as memory placement for massive thread-based systems. Data locality and affinity are key
concepts to exploit locality and the memory capabilities for these next-generation of systems. 
In this paper, we evaluate the latest affinity features of OpenMP 4.5 using Oak Ridge National Laboratory mini-apps 
and test them on a system with two Power8 and K40 NVIDIAs. We experiment with the different affinity settings provided by OpenMP 4.5 and measure their effects via hardware counters available on the POWER8 system to measure affinity at scale. Based on this study we describe the current state of art, the challenges we faced in quantifying effects of affinity, inadequacy in the current OpenMP programming model and how some of these may be addressed in OpenMP 5.0.
% Importance of affinity, it is well understood for traditional systems but lil understanding on massively multi-threaded heterogeneous architectures (including the different memories). OpenMP implicitly controls affinity but there is no explicit way to control memory hierarchy.
\end{abstract}

\section{Introduction}
% How affinity works on 4.0 and 4.5
% What discussions are being proposed for 5.0 --telecon
\label{sec:intro}
The effects of affinity is a widely studied problem. Most programming models 
take advantage of the architecture and data-access patterns by providing some 
implicit or explicit control over data and process/thread placement. For example,
Unified Parallel C (UPC), a Partitioned Global Address Space (PGAS) language 
provides a \textbf{shared} qualifier to distinguish between data 
accessible to all the UPC \textit{threads} vs. private data. The \textit{shared} scalar data has 
affinity to thread 0, while arrays can have affinity granularity of \textit{cyclic, blocked-cyclic ,}
and \textit{blocked}. These are chosen by the application programmer based on the 
knowledge of the data access patterns within the application. 

Similarly for OpenMP, there are 
different ways to specify and affect affinity of data, threads, and work units or \textit{tasks}. 
The new OpenMP 4.5 release provides a substantial 
improvement on the support for programming of accelerator and GPU devices. 
Amongst the new features introduced are, support for parallelization of loops with 
well-structured dependencies, mechanisms for unstructured data mapping and 
asynchronous execution, support to divide loops into tasks without requiring all 
threads to execute the loop, reductions for C\/C++ arrays, a new hint mechanisms to 
provide guidance on the relative priority of tasks and on preferred synchronization 
implementations, SIMD extensions, improved support for Fortran 2003, and
thread affinity support through runtime functions to determine the effect of thread 
affinity clauses. In this paper we focus on the affinity aspect of OpenMP with respect 
to the emerging OpenPOWER systems.

\subsection{Memory Binding}
Most systems provide implicit data affinity control through policies like \textit{first-touch} 
and \textit{next-touch}. First-touch is more appropriate for applications where the 
first access to data is representative of the application\'s data accesses throughout 
the life of the application. This policy has been adopted as default on many systems. 
For applications that have a more dynamic access pattern, the \textit{next-touch} 
policy may be more appropriate. Here the data is marked to be placed on the node of the 
next CPU that accesses it. For OpenMP, \textit{first-touch} translates to data being 
placed near the thread that first accesses it. Even without any other affinity mechanism this 
can cause significant impact, for e.g., if the data is initialized by thread 0 only, but later is accessed 
by all the threads, the \textit{first-touch} policy would locate memory on the node where 
thread 0 is placed thus resulting in high cost accesses for threads not placed on the same node. 

\subsection{Thread Placement}
OpenMP provides OMP\_PROC\_BIND ICV to set the thread affinity policy. The legitimate value for 
this environment variable is either true, false, or a comma separated list of master, close, or spread. 
When the values are specified in a list, they correspond to the thread affinity policy to be used for 
parallel regions at the corresponding nested level. In combination with the OMP\_PLACES ICV, 
users may have complete control on the thread affinity and their placement on a given hardware. 
OMP\_PLACES ICV can be one of two types of values: either an abstract name describing a set 
of places or an explicit list of places described by non-negative numbers. Pre-defined abstract 
names include \textit{threads, cores,} and \textit{sockets}. When expressed as numbers, places 
represent the smallest unit of execution exposed by the execution environment, which is typically 
a hardware thread.

In conjunction to places represented by non-negative numbers, intervals is another handy way to 
express \textit{places} in OpenMP. They are specified using the \textit{$<$lower-bound$>$ : $<$length$>$ : $<$stride$>$} notation. For example, a user could specify exact CPUs to place the OpenMP threads or a range of CPUs based on the application characteristics to best utilize the underlying hardware.


\subsection{POWER8 System}
%Needs to be paraphrased 
The POWER8 processor is the latest RISC (Reduced Instruction Set Computer) microprocessor from IBM and the first processor supporting the new OpenPOWER software environment. 

\begin{figure}
    \centering
    \begin{subfigure}[b]{0.4\textwidth}
         {\includegraphics[width=1.0\textwidth]{./Images/P8.pdf}}
%  	\vspace{-0.0pc}
	 \caption{Power8 processor chip.~\cite{IBM_P8}}
 	 \label{fig:p8_1}
    \end{subfigure}
     \centering
    \begin{subfigure}[b]{0.4\textwidth}
         {\includegraphics[height=0.7\textwidth]{./Images/P8_memory.pdf}}
  %	\vspace{-0.0pc}
	 \caption{Power8 Memory Access.~\cite{IBM_P8}}
	  \label{fig:p8_2}
    \end{subfigure}
  \caption{POWER8 Overview}\label{fig:POWER8}
\end{figure}







 

%\section{Motivation}
%How do affinity constructs work and see their adequacy and propose extensions ---on OpenPOWER 
%\label{sec:motivation}
%The new Exascale system at ORNL, Summit, will be an OpenPOWER system with NVIDIA GPUs. To provide a better understand of the working of OpenMP 
programs on this novel architecture we look at the most impactful aspects of the programming model. By examining the POWER8 hardware counters while testing the OpenMP 4.0 affinity features implemented in different compilers available on the new system, we hope to measure effects of affinity at scale. Based on this study, we describe the challenges that we faced in quantifying effects of affinity, inadequacy in the current OpenMP programming model and how some of these may be addressed in OpenMP 5.0. We also endeavor to find the optimum OpenMP settings for different application kernels that can yield the best performance.

\section{Motivation}
\label{sec:method}
We look closely at the different hardware counters available on the new POWER8 system to better understand the effects of data affinity and thread placement. Performance instrumentation in POWER8 is provided in two layers: the \textbf{Core Level Performance Monitoring} (CLPM) and the \textbf{Nest Level Performance Monitoring} (NLPM). CLPM allows for monitoring of the core pipeline efficiency of the front-end, branch prediction, schedulers etc., along with behavioral metrics such as stalls, execution rates, thread prioritization and resource sharing, and utilizations of resources etc. On the other hand NLPM provides a way to instrument the L3 cache, interconnect fabric and memory channels/controllers. POWER8 has an enhanced Cycles Per Instruction (CPI) Accounting Model. The POWER8 CPI Stack accounts for stalled, waiting to complete, thread blocked, completion table empty, completion and other miscellaneous cycles. The stalled cycles are further classified based on the cause of the stall. Newly added to this group for the POWER8 architecture is the finer granularity of \textit{ Stall cycles due to Dcache Misses}. Since we want to prevent cycles wasted due to data misses, we focus on the sub-set of hardware counters mentioned in table~\ref{tab:hwct}. Here we focus on the \texxtbf{data cache} stall cycles to better understand the overhead of using different \textit{bind} and \textit{place} configurations. The collection of the counter values are enabled by a system provided script. This allows for access to counters that may not be represented as literary strings and accessible via other application profiling tools. 

\begin{table*}[t]
\vspace{-0.5pc}
\caption{POWER8 Relevant Counters}
\centering
\begin{tabular} { | c | c | }
\hline
{\bf Hardware Counter} & {\bf Description}  \\ \hline
PM\_CMPLU\_STALL\_DCACHE\_MISS & Stall cycles due to data cache misses	\\	\hline
PM\_CMPLU\_STALL\_DMISS\_REMOTE & Stall cycles due to data cache miss on remote \\	\hline
PM\_CMPLU\_STALL\_DMISS\_DISTANT & Stall cycles due to data cache miss on distant \\ \hline
\end{tabular}
\label{tab:hwct}
%\vspace{-0.0pc}
\end{table*}
 %methodology, implementation and results here

\section{Results}
\label{sec:results}
As explained in Section~\ref{sec:intro} OpenMP 4.0 / 4.5 provides two environment variables OMP\_PROC\_BIND and OMP\_PLACES to help users define the thread placement and bindings for their OpenMP application which we refer to as \textit{OpenMP Affinity}.%
%\begin{figure}[h!]
%  \centering
%  \includegraphics[height=0.4\textwidth, width=0.8\textwidth]{./Images/10Perf.pdf}
%       \caption{Performance with 10 OpenMP threads on \textit{}}
%       \label{fig:10th}
%\end{figure}
%
 We experiment with two version of the Jacobi program: a version that is optimized for data locality via correct memory placement using the first touch policy and another version where 
 all the data is close to where thread 0 is.   We use a jacobi problem size of (60000 X 60000)  to make to stress the memory subsystem, specifically to utilize the entire L3 and L4 caches.
 
 We run both programs with 10, 20, 40, 80 and 160  number of threads threads with different OpenMP affinity settings and record the POWER8 hardware counters.
 Figure~\ref{fig:20th} shows the performance of 20 OpenMP threads with different OpenMP Affinity settings. We observed that after 20 OpenMP threads the speedup does not vary significantly. The best speedup and efficiency combination (16 speedup, 80\% efficinecy) is achieved with 20 OpenMP threads with the OpenMP Affinity configuration of \textit{(spread, threads)}.
%
\begin{figure}[h!]
  \centering
  \includegraphics[height=0.4\textwidth, width=0.95\textwidth]{./Images/20Perf.pdf}
       \caption{Performance with 20 OpenMP threads on a two POWER8 socket system for the two versions of the Jacobi program.}
       \label{fig:20th}
\end{figure}
%
\begin{figure}[h!]
  \centering
  \includegraphics[height=1.4\textwidth, width=0.95\textwidth]{./Images/ImpAllV.pdf}
       \caption{Comparing Performance Improvement between the two version of the Jacobi program using 
       different number of OpenMP threads and thread affinity settings on a two POWER8 socket system}
       \label{fig:imp}
\end{figure}

Figure~\ref{fig:imp} shows the improvement of the locality-aware optimized version (with good data placement) of the Jacobi program over the unoptimized locality-unaware (all data local to thread 0) version when using different OpenMP Affinity settings for different OpenMP thread counts. 
We observe that for the \textit{(master, threads)} OpenMP affinity configuration all threads execute on a single hardware thread (CPUID 0).
 
When using \textit{(master, core)}, all threads execute on the different hardware threads that belong to the core where the master thread is running (the P8 processor has eight hardware threads per core), similarly for \textit{(master, socket)} all threads execute in the hardware threads of the socket where the master thread is running. In this case all threads will be bind to any of the CPU ids from 0 to 79. When OMP\_PROC\_BIND set to master we see in Figure ~\ref{fig:imp} that there is no improvement on the locality aware over the non-locality aware versions (e.g. using first-touch policy) because all the threads are running using the 
same NUMA domains. For \textit{(close, threads)}, we observe that all OpenMP threads run on hardware threads close to each other (on CPU ids: 0-19). 
All of these cases don't suffer from memory locality issues because they access memory memory that belong to the same NUMA domain or memory that is local to the socket (e.g. two NUMA domains per socket).
In the \textit{(spread, sockets)} and \textit{(close, sockets)} configuration threads are spread across sockets but may be mapped to hardware threads running on same core. We observed that the \textit{(true, threads)} configuration is equivalent to the \textit{(close,threads)} according to the GCC OpenMP runtime thread mappings.
For all OpenMP affinity settings  with OMP\_PROC\_BIND set to false, threads can migrate and are not bound to a specific thread, core or socket. This migration makes it less impactful on the data placement, but suffers from degraded performance.

From the above discussion it is clear that not all OpenMP Affinity configurations are equal, moreover currently it lacks the ability to specify affinity based on NUMA \textbf{and} NUCA domains of emerging architectures like POWER8. This is the first step in understanding the need for new OpenMP affinity features to successfully deploying OpenMP on POWER machines. %Need something more here. 
\begin{figure}[h!]
  \centering
  \includegraphics[height=0.4\textwidth, width=0.95\textwidth]{./Images/HW.pdf}
       \caption{Comparing Hardware Counter Change}
       \label{fig:HW}
\end{figure}
%
%Hardware Counter notes
Next we look at the hardware counters on the POWER8 system corresponding to the different configurations to explain the improvement we observed in Figure~\ref{fig:imp}.

%to see what hardware counters could potentially be used as indicators to signify a OpenMP Affinity choice for a given problem. 
We select the three cases of OpenMP Affinity tuple that represent the best, mid, and worst improvement as seen in Figure~\ref{fig:imp}. 
%the cases where we see more improvement for data locality on a given affinity setting 
Selected cases are \textit{(spread, sockets)},  \textit{(spread, cores)}, and \textit{(master, thread)}. 
For these cases we record the hardware counters for the locality aware and locality unaware versions and calculate the improvement as the value of their difference as a percentage of the value of the locality unaware hardware counter value. 
From Figure~\ref{fig:HW} we see that the two hardware counters that show the effects of data locality the most are DMISS\_DISTANT, DMISS\_L3MISS.
We would have expected to see more significant variation in the value of DMISS\_REMOTE, but we found that in some cases, these remote accesses can be cached.
For example, the case of \textit{(spread, sockets)} has better DMISS\_REMOTE improvement than  \textit{(spread, cores)} which is counterintuitive. 
This is because in \textit{(spread, sockets)} some threads (not all) are running on the same core sharing local cache lines for (L1, L2) and thus taking advantage of cache reuse for remote data access. 
This can also be seen by the significant improvement in DMISS\_DISTANT which quantifies the stalls by L1 reloads from distant interventions and memory. 

The improvements we see in \textit{(spread, cores)} are more due to DMISS\_L21\_L31, which shows a better utilization of the L2/L3 cache as this hardware counter measures the stall cycles by Dcache miss which are resolved on chip. 
In the case of  \textit{(spread, cores)} we are effectively increasing the amount of 
L2 cache available to each OpenMP thread, as each thread has access to its own L2 cache on a given core. 
For the \textit{(master, thread)} case, there is very little improvement in the memory subsystem utilization as everything is running on the same thread (or CPU id) and the most of the data is local to the socket. In this case the data-locality version does not make any difference because all the threads are time-sharing the same hardware thread. This is also true for the case  \textit{(close, threads)} where
 we don't see improvements on the data locality aware version since data is local to the threads. 
%%Option 1
%\begin{figure}[h!]
%  \centering
%  \includegraphics[height=0.4\textwidth, width=0.95\textwidth]{./Images/AllHW1.pdf}
%       \caption{Comparing Hardware Counters for Locality Aware Jacobi Program  Executions}
%       \label{fig:HW1}
%\end{figure}
%%Option 2
%\begin{figure}[h!]
%  \centering
%  \includegraphics[height=1\textwidth, width=0.95\textwidth]{./Images/AllHW2.pdf}
%       \caption{Comparing Hardware Counters for Locality Aware Jacobi Program  Executions}
%       \label{fig:HW2}
%\end{figure}
%\begin{figure}[h!]
%  \centering
%  \includegraphics[height=0.4\textwidth, width=0.95\textwidth]{./Images/HW_RD.pdf}
%       \caption{Comparing Hardware Counters}
%       \label{fig:HW3}
%\end{figure}
%Figure~\ref{fig:HW3} shows the values of the stall cycles for the hardware counters discussed in Table~\ref{tab:hwct} and Table~\ref{tab:cl}. Though the individual values are not significant, the figure gives a clear idea of the relative difference in the hardware counter values for all locality aware runs. 

POWER8 provides these unique set of hardware counters to distinguish OpenMP configurations that have less useless cycles on the memory sub-system. Specifically, DMISS\_REMOTE and, more importantly, DMISS\_DISTANT are key in identifying if the program suffers from bad data locality as indicated in Figure~\ref{fig:imp}.
%For example, from the data collected for 20 OpenMP threads \textit{(spread, threads)},  \textit{(close, cores)}, \textit{(true, cores)},  \textit{(close, sockets)}, \textit{(spread, sockets)} have similar high speed-ups but the PM\_CMPLU\_STALL\_DMISS\_L2L3 is least in the \textit{(spread, sockets)} configuration. 


 

\section{Related Work}
\label{sec:related}
\input{text/related}

\section{Conclusions and Future Work}
\label{sec:conclusion}
In this paper we evaluated how OpenMP supports the concept of affinity as well as memory placement on on the POWER8 architecture. 
Data locality and affinity are key concepts to exploit the compute and memory capabilities to achieve good performance by minimizing data motion across NUMA domains. 
The main contribution of this paper is to evaluate current affinity features of OpenMP 4.0 on the POWER8 processors, and on how to measure its 
effects on data locality on a system with two P8 sockets. What we notice from the experiments the effect of data placement and data locality is dependent on how threads 
are mapped to the architectures. In some OpenMP affinity test cases, we show that the POWER8 architectures using its NUCA L3 caches can hide the cost of accessing 
remote memory (e.g, as shown in the experiment (spread, cores) when running a thread per core since maximizes local caches are available per thread. On other cases, 
when threads share some of the cores, there is a benefit of cache reuse in the non-shared L2 and L1 caches. Improving data locality in the application. In this paper we show that optimizing an application for data locality, the improvements will depend on the kind of affinity used. 

Future version of OpenMP affinity model need to support better the concept of NUMA domains. This is possibly an extension that can be added
to OpenMP 5.0 via the \emph{OMP\_PLACES} so that threads can be mapped more efficiently to NUMA domains. Another type of extensions is to integrate
the concept of OMP\_PLACES with the OpenMP target directive and device\_num. It would be great if we could map OpenMP \emph{target} and \emph{target data} 
regions to NUMA domains to control data and thread placement.



% conference papers do not normally have an appendix

%\section{Acknowledgments}
%\input{ack}

\bibliographystyle{splncs}
\bibliography{references}

\end{document}
